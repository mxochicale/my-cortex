\documentclass[modern]{aastex62}

% Load the corTeX style definitions
\input{cortex}

% Bibliography stuff
\bibliographystyle{aasjournal}

% Begin!
\begin{document}

% Title
\title{\textsc{My corTeX}: Continuous Integration for Scientific Articles}

% Author list
\author[0000-0002-8225-7517]{Miguel Xochicale}
\email{miguel@xmail.com}
\affil{Center~for~Computational~Astrophysics, London, UK}


% Introduction
\section{The Long and Short of it}
\label{sec:intro}
%
\textsc{corTeX} is a very simple \textsf{GitHub} repository aimed at making
it easy to make your scientific articles open source and reproducible.
\textsc{corTeX} currently does two things. First, it defines the
\textsf{\textbackslash oscaption} command, which adds
links to all figure captions pointing to the exact script that generated it;
see Figure~\ref{fig:pretty_function}.
%
\begin{figure}[h!]
    \begin{centering}
    \includegraphics[width=0.5\linewidth]{figures/pretty_function.pdf}
    \oscaption{pretty_function}{%
        This is a plot of a pretty function. And at the end of this
        caption is a symbol with a link to the \emph{exact} script
        that generated it, hosted on \textsf{GitHub}.
        \label{fig:pretty_function}
    }
    \end{centering}
\end{figure}
%

Second, it defines the \textsf{\textbackslash proof} environment, which
adds a link to any equation pointing to a proof, derivation, or
verification of the equation in the form of a \textsf{Jupyter} notebook:
%
\begin{proof}{euler}
    \label{eq:euler}
    e^{i\pi} + 1 = 0.
\end{proof}
%
Check out \citet{Luger2018} for an example of a paper that uses these
features.

% Bibliography
\pagebreak
\bibliography{bib}

\end{document}
